\documentclass[12pt]{article}

\RequirePackage{amsmath}
\RequirePackage{amsthm}
\RequirePackage{amssymb}
\RequirePackage[mathscr]{eucal}
\RequirePackage{mathtools}
\RequirePackage{etoolbox}

\usepackage[red]{zhoucx-notation}

% correct bad hyphenation here
\hyphenation{op-tical net-works semi-conduc-tor}

\renewcommand{\qedsymbol}{\hfill\rule{2mm}{2mm}}

\pagestyle{fancy}
\fancyhf{}
\setlength{\headheight}{15pt}
\rhead{\textsf{Chapter 10, Model Assessment and Selection}}
\lhead{\textsf{Chenxi Zhou}}
\renewcommand{\headrulewidth}{1pt}
\cfoot{\thepage}

\newcommand{\titlebox}[4]{
\begin{tcolorbox}[colback = blue!5!white!95, colframe = blue!70!black
% colback = yellow!30!white, colframe = yellow!70!black 
]
  \noindent \textbf{ #1 } \hfill \textit{#2} 
  \begin{center}
  	 \LARGE{\textbf{#3}}
  \end{center}
\textbf{Chapter:} \textit{#4} \hfill \textbf{Prepared by:} \textit{Chenxi Zhou}
\end{tcolorbox}
}

\begin{document}

\titlebox{Notes on Statistical and Machine Learning}{}{Model Assessment and Selection}{10}
\thispagestyle{plain}

\vspace{10pt}

This note is prepared based on \textit{Chapter 7, Model Assessment and Selection} in \textcite{Friedman2001-np}. In this chapter, we study 
\begin{enumerate}
	\item how to \emph{assess} the performance, in particular, the generalized performance, of a model, and 
	\item how to use these assessment to \emph{select} model. 
\end{enumerate}


\section*{I. Bias, Variance and Model Complexity} 

\begin{enumerate}[label=\textbf{\arabic*.}]

	\item \textbf{Generalization Performance:} The \textit{generalization performance} of a learning method relates to its prediction capability on an \emph{independent test data}. It is important since it guides the choice of learning method or model and gives a measure of the quality of the ultimately chosen model. 
	
	\item \textbf{Basic Setup:} 
	\begin{enumerate}
		\item a target variable $Y$ (assumed to be \emph{continuous} for the moment), 
		\item a vector of inputs $X$, 
		\item a prediction model $\hat{f}$ estimated from a training dataset $\train$, 
		\item a loss function $L \parens{Y, \hat{f} \parens{X}}$, measuring errors between $Y$ and $\hat{f} \parens{X}$. Typical choices are 
		\begin{itemize}
			\item \textit{Squared Error:} $L \parens{Y, \hat{f} \parens{X}} = \parens{Y - \hat{f} \parens{X}}^2$; 
			\item \textit{Absolute Error:} $L \parens{Y, \hat{f} \parens{X}} = \abs{Y - \hat{f} \parens{X}}$. 
		\end{itemize}
	\end{enumerate}
	
	\item \textbf{Test/Generalization Error:} \textit{Test error} or \textit{generalization error} is the prediction error over an \underline{independent test sample}
	\begin{align}\label{eq-test-error}
		\mathrm{Err}_{\train} := \E \bracks{ L \parens{Y, \hat{f} \parens{X}} \,\vert\, \train}
	\end{align}
	where both $X$ and $Y$ are drawn randomly from their joint distribution (population). 
	
	\textit{Remark.} In \eqref{eq-test-error}, the training set $\train$ is fixed, and test error refers to the error for this \textit{specific} training set. 
	
	\item \textbf{Expected Prediction Error:} The \emph{expected prediction error} (or \emph{expected test error}) is 
	\begin{align}\label{eq-epe}
		\mathrm{Err} := \E \bracks{ L \parens{Y, \hat{f} \parens{X} } } = \E \bracks{ \mathrm{Err}_{\train} }
	\end{align}
	
	\textit{Remark 1.} This expectation averages over \textit{everything} that is random, including the randomness in the \underline{training set $\train$} that produced $\hat{f}$. 
	
	\textit{Remark 2.} Estimation of $\mathrm{Err}_{\calT}$ is the \emph{goal}, but $\mathrm{Err}$ is more amenable to statistical analysis. 
	
	\item \textbf{Training Error:} \emph{Training error} is the average loss over the \underline{training sample}
	\begin{align}
		\overline{\mathrm{err}} := \frac{1}{n} \sum_{i=1}^n L \parens{y_i, \hat{f} \parens{\bx_i}}. 
	\end{align}

	\textit{Remark.} If the model we build is getting increasingly complex, we use the training dataset more to exploit the complicated underlying structures. Then, 
	\begin{enumerate}
		\item there is a tendency of lowering the bias and increasing the variance. There is some \emph{intermediate} model complexity that gives the minimum expected test error; 
		\item training error consistently decreases with model complexity and can drop to 0 if one increases the model complexity sufficiently. However, a model with 0 training error is \emph{overfit} to the training set and is typically generalized \emph{poorly}. 
	\end{enumerate}
	
	\item \textbf{Analogous Results for Categorical Response Variables:} Suppose the response variable is qualitative or categorical, denoted by $G$, that takes on one of $W$ values in $\calW := \sets{1, \cdots, W}$. We model the probabilities 
	\begin{align*}
		p_w \parens{X} := \Pr \parens{G = w \,\vert\, X}, 
	\end{align*}
	and classify $X$ according to the following rule 
	\begin{align*}
		\widehat{G} \parens{X} = \argmax_{w \in \calW} \hat{p}_w \parens{X}. 
	\end{align*}
	Some typical choices of loss functions are 
	\begin{itemize}
		\item \textit{0-1 loss:} 
		\begin{align*}
			L \parens{G, \widehat{G} \parens{X}} = \indic \parens{ G \ne \widehat{G} \parens{X}}; 
		\end{align*}
		\item \textit{$-2 \times$Log-likelihood loss:} 
		\begin{align*}
			L \parens{Y, \hat{p} \parens{X}} = & \, -2 \sum_{w=1}^W \indic \parens{G = w} \log \hat{p}_w \parens{X} \\ 
			= & \, -2 \log \hat{p}_G \parens{X}. 
		\end{align*}
		The quantity ``$-2\times$log-likelihood'' is sometimes referred to the \textit{deviance}. 
	\end{itemize}
	For categorical variables, similar to before, 
	\begin{enumerate}
		\item the \textit{test error}, $\mathrm{Err}_{\train}$, is defined to be 
		\begin{align*}
			\mathrm{Err}_{\train} = \E \bracks[\big]{L \parens{G, \widehat{G} \parens{X}} \,\vert\, \train}, 
		\end{align*}
		the population misclassification error of the classifier trained on $\train$; 
		\item the \textit{expected prediction error}, $\mathrm{Err}$, is the expected misclassification error 
		\begin{align*}
			\mathrm{Err} = \E \bracks{\mathrm{Err}_{\train}}; 
		\end{align*}
		\item the \textit{training error} is 
		\begin{align*}
			\overline{\mathrm{err}} = \frac{1}{n} \sum_{i=1}^n L \parens{g_i, \widehat{G} \parens{\bx_i}}, 
		\end{align*}
		where $\sets{\parens{\bx_i, g_i}}_{i=1}^n$ is the training dataset and $g_i \in \calW$ is the observed class label for the $i$-th observation for all $i = 1, \cdots, n$. 
	\end{enumerate}
	
	\item \textbf{Log-Likelihood as a Loss Function:} The log-likelihood function can be used a loss function for general response density functions, such as Poisson, gamma, exponential, log-normal and so on. If $\Pr_{\theta \parens{X}}$ is the density function of $Y$, where $\theta \parens{X}$ is the parameter depending on the predictor $X$, then the loss function is 
	\begin{align*}
		L \parens{Y, \theta \parens{X}} = -2 \cdot \log \Pr_{\theta \parens{X}} \parens{Y}. 
	\end{align*}
	
	\textit{Remark.} The ``$-2$'' in the front is to make the log-likelihood loss for the Gaussian distribution match the squared-error loss function. 
	
	\item \textbf{Model Selection vs. Model Assessment:} Typically, the model has a tuning parameter(s) $\alpha$ that affects the model complexity, and the predictions can be written as $\hat{f}_{\alpha} \parens{\bx}$. We wish to determine the value of $\alpha$ minimizing the errors. 
	
	In this procedure, there are two separate goals: 
	\begin{itemize}
		\item \textit{Model selection:} estimating the performance of different models in order to choose the best one; 
		\item \textit{Model assessment/evaluation:} having chosen a final model, estimating its prediction error (generalization error) on new data. 
	\end{itemize}
	If we have a sufficiently rich dataset, one can randomly split it into \emph{three} parts: 
	\begin{itemize}
		\item \textit{Training Set:} used to fit the models; 
		\item \textit{Validation Set:} used to estimate prediction error for model selection; 
		\item  \textit{Test Set:} used for the assessment of the generalization error of the final chosen model. This set should be brought out \emph{only} at the end of the data analysis. 
	\end{itemize}
	A typical split for three parts might be $50\%$ for training, $25\%$ each for validation and testing. 
	
\end{enumerate}


\section*{II. The Bias-Variance Decomposition} 

\begin{enumerate}[label=\textbf{\arabic*.}]

	\item \textbf{The Bias-Variance Decomposition --- General Case:} Assume that the model is of the form 
	\begin{align*}
		Y = f \parens{X} + \varepsilon, 
	\end{align*}
	where we assume $\E \bracks{\varepsilon} = 0$ and $\var \bracks{\varepsilon} = \sigma_{\varepsilon}^2$. Then, the \emph{expected prediction error} of a regression fit $\hat{f}$ at $X = \bx_0$ using the squared-error loss is derived as follows: 
	\begin{align*}
		\mathrm{Err} \parens{\bx_0} = & \, \E \bracks[\big]{\parens{Y - \hat{f}\parens{\bx_0} }^2 \bigm| X = \bx_0 } \\
		= & \, \E \bracks[\big]{\parens{Y - f \parens{\bx_0} + f \parens{\bx_0} - \hat{f} \parens{\bx_0}}^2 \bigm| X = \bx_0 } \\
		= & \E \bracks[\big]{ \parens{Y - f \parens{\bx_0}}^2 \bigm| X = \bx_0 } - 2 \E \bracks[\big]{ \parens{Y - f \parens{\bx_0}} \parens{ f \parens{\bx_0} - \hat{f} \parens{\bx_0}} \bigm| X = \bx_0 } \\ 
		& \, + \E \bracks[\big]{ \parens{ f \parens{\bx_0} - \hat{f} \parens{\bx_0}}^2 \bigm| X = \bx_0 } \\
		= & \sigma_{\varepsilon}^2 + 0 + \E \bracks[\big]{ \parens{ f \parens{\bx_0} - \hat{f} \parens{\bx_0}}^2 \bigm| X = \bx_0 } \\ 
		= & \sigma_{\varepsilon}^2 + \E \bracks[\big]{ \parens{ f \parens{\bx_0} - \E \bracks{\hat{f} \parens{\bx_0}} + \E \bracks{\hat{f} \parens{\bx_0}} - \hat{f} \parens{\bx_0}}^2 \bigm| X = \bx_0 } \\ 
		= & \sigma_{\varepsilon}^2 + \E \bracks[\big]{ \parens{ f \parens{\bx_0} - \E \bracks{\hat{f} \parens{\bx_0}}}^2 \bigm| X = \bx_0 } + \E \bracks[\big]{\parens{ \E \bracks{\hat{f}\parens{\bx_0}} - \hat{f} \parens{\bx_0}}^2 \bigm| X = \bx_0 } \\
		= & \text{Irreducible Error} + \text{Bias}^2\parens{\hat{f}\parens{\bx_0}} + \var \bracks{\hat{f}\parens{\bx_0}}. 
	\end{align*}
	Analysis of the three terms: 
	\begin{itemize}
		\item The \textit{first term} is the \underline{variance of the target around its true mean} $ f \parens{\bx_0} $, and can \underline{not} be avoided no matter how well we estimate $ f \parens{\bx_0} $, unless $\sigma_{\varepsilon}^2 = 0$; 
		\item The \textit{second term} is the \underline{squared bias}, the amount by which the average of our estimate $\hat{f} \parens{\bx_0}$ differs from the true mean $f \parens{\bx_0}$; 
		\item The \textit{last term} is the \underline{variance}, the expected squared deviation of $\hat{f} \parens{\bx_0}$ around its mean. 
	\end{itemize} 
	\textit{Remark.} Typically, the more complex we make the model, the \textit{lower} the (squared) bias but the \textit{higher} the variance. 
	
	\item \textbf{The Bias-Variance Decomposition for $k$-Nearest Neighbor Regression:} For the $k$-nearest-neighbor regression fit, the bias-variance decomposition has the form 
	\begin{align*}
		\mathrm{Err} \parens{\bx_0} = & \, \E \bracks[\big]{ \parens{Y - \hat{f}_k \parens{\bx_0}}^2 \,\vert\, X = \bx_0 } \nonumber \\
		= & \, \sigma_{\varepsilon}^2 + \bracks[\bigg]{ f \parens{\bx_0} - \frac{1}{k} \sum_{l=1}^k f\parens{\bx_{ \parens{l} }} }^2 + \frac{\sigma_{\varepsilon}^2}{k}, 
	\end{align*}
	where we assume the training inputs $\bx_i$ are fixed and the randomness arises from $y_i$'s. 
	
	\textit{Remark.} The number of neighbors $k$ is \emph{inversely} related to the model complexity. 
	\begin{itemize}
		\item For small $k$, the estimate $ \hat{f} \parens{\bx}$ can potentially adapt itself better to the underlying $f \parens{\bx}$; i.e., a smaller bias; 
		\item As one increases $k$, the bias will typically increase and the variance decreases. 
	\end{itemize}
	
	\item \textbf{The Bias-Variance Decomposition for Linear Regression Model:} Recall the linear model fit is $\hat{f} \parens{\bx} = \bx^\top \widehat{\bbeta}$, where $\widehat{\bbeta} \in \Real^p$ is the parameter estimated by the least squares method. Then, 
	\begin{align}
		\mathrm{Err} \parens{\bx_0} = & \, \E \bracks[\big]{ \parens{Y - \hat{f} \parens{\bx_0}}^2 \,\vert\, X = \bx_0 } \nonumber \\
		= & \, \sigma_{\varepsilon}^2 + \E \bracks[\big]{ f \parens{\bx_0} - \E \bracks{\hat{f} \parens{\bx_0}} }^2 + \sigma_{\varepsilon}^2 \cdot \norm{\bh \parens{\bx_0}}_2^2, \label{eq-test-err-linear-reg}
	\end{align}
	where we assume the design matrix $\bX$ is of the full rank and $\bh\parens{\bx_0} = \bX\parens{\bX^\top \bX}^{-1} \bx_0$. 
	
	We show $\var \bracks{\hat{f} \parens{\bx_0}} = \sigma_{\varepsilon}^2 \cdot \norm{\bh \parens{\bx_0}}^2$. Recall that the least squares regression fit in the full rank case is of the form 
	\begin{align*}
		\hat{f} \parens{\bx_0} = \bx_0^\top \widehat{\bbeta} = \bx_0^\top \parens{\bX^\top \bX}^{-1} \bX^\top \bY.  
	\end{align*}
	Then, its variance is
	\begin{align*}
		\var \bracks{\hat{f} \parens{\bx_0}} = & \, \var \bracks{ \bx_0^\top \parens{\bX^\top \bX}^{-1} \bX^\top \bY } \\ 
		= & \, \bx_0^\top \parens{\bX^\top \bX}^{-1} \bX^\top \var \bracks{\bY} \parens{\bx_0^\top \parens{\bX^\top \bX}^{-1}\bX^\top}^\top \\
		= & \, \bx_0^\top \parens{\bX^\top \bX}^{-1} \bX^\top \cdot \sigma_{\varepsilon}^2 \bI \cdot \bX \parens{\bX^\top \bX} \bx_0 \\
		= & \, \sigma_{\varepsilon}^2 \cdot \norm{ \bX \parens{\bX^\top \bX}^{-1} \bx_0}_2^2 \\ 
		= & \, \sigma_{\varepsilon}^2 \cdot \norm{\bh\parens{\bx_0}}_2^2. 
	\end{align*}
	
	Replacing $\bx_0$ by $\bx_i$ for all $i = 1, \cdots, n$ and taking the average, we have 
	\begin{align*}
		\frac{1}{n} \sum_{i=1}^n \var \bracks{\hat{f} \parens{\bx_i}} = & \, \frac{1}{n} \sum_{i=1}^{n} \sigma_{\varepsilon}^2 \norm{\bX \parens{\bX^\top \bX}^{-1} \bx_i}^2 \\ 
		= & \, \frac{\sigma_{\varepsilon}^2}{n} \sum_{i=1}^n \bx_i^\top \parens{\bX^\top \bX}^{-1} \bX^\top \bX \parens{\bX^\top \bX}^{-1} \bx_i \\ 
		= & \, \frac{\sigma_{\varepsilon}^2}{n} \sum_{i=1}^n \bx_i^\top \parens{\bX^\top \bX}^{-1} \bx_i \\ 
		= & \, \frac{\sigma_{\varepsilon}^2}{n} \sum_{i=1}^n \tr \parens{ \bx_i^\top \parens{\bX^\top \bX}^{-1} \bx_i} \\ 
		= & \, \frac{\sigma_{\varepsilon}^2}{n} \sum_{i=1}^n \tr \parens{ \bx_i \bx_i^\top \parens{\bX^\top \bX}^{-1}} \\ 
		= & \, \frac{\sigma_{\varepsilon}^2}{n} \tr \parens[\Bigg]{\parens[\bigg]{\sum_{i=1}^n \bx_i \bx_i^\top} \parens{\bX^\top \bX}^{-1}} \\ 
		= & \, \frac{\sigma_{\varepsilon}^2}{n} \tr \parens[\big]{\parens{\bX^\top \bX} \parens{\bX^\top \bX}^{-1}} \\ 
		= & \, \frac{\sigma_{\varepsilon}^2}{n} \tr \parens{\bI_p} \\ 
		= & \, \frac{p \sigma_{\varepsilon}^2}{n}. 
	\end{align*}
	Therefore, the \textit{in-sample error} is 
	\begin{align*}
		\overline{\mathrm{err}} = \frac{1}{n} \sum_{i=1}^n \mathrm{Err} \parens{\bx_i} = \sigma_{\varepsilon}^2 + \frac{1}{n} \sum_{i=1}^n \parens[\big]{ f \parens{\bx_i} - \E \bracks{ \hat{f} \parens{\bx_i} } }^2 + \frac{p}{n} \sigma_{\varepsilon}^2. 
	\end{align*}
	Notice that the complexity is directly related to the number of parameters $p$. 
	
	\item \textbf{The Bias-Variance Decomposition for Ridge Regression:} Consider the ridge regression where we solve the following optimization problem 
	\begin{align*}
		\minimize_{\bbeta} \ \braces[\bigg]{ \parens{\bY - \bX \bbeta}^\top \parens{\bY - \bX \bbeta} + \lambda \bbeta^\top \bbeta }, 
	\end{align*}
	where $\lambda > 0$ is the tuning parameter and we assume there is no intercept term. The minimizer to the preceding optimization problem, denoted by $\widehat{\bbeta}_{\lambda}$, is 
	\begin{align*}
		\widehat{\bbeta}_{\lambda} = \parens{\bX^\top \bX + \lambda \bI_p}^{-1} \bX^\top \bY. 
	\end{align*}
	Let the prediction function be $\hat{f}_{\lambda} \parens{\bx} = \bx^\top \widehat{\bbeta}_{\lambda}$. 
	
	The test error has the same form as in \eqref{eq-test-err-linear-reg}, but the bias and variance components are different: 
	\begin{itemize}
		\item \textit{Bias:} Let $\bbeta^*$ be the parameters of the best-fitting linear approximation to $f$, i.e., 
		\begin{align*}
			\bbeta^* := \argmin_{\bbeta} \E \bracks[\big]{ \parens{f \parens{X} - X^\top \bbeta}^2 }, 
		\end{align*}
		where the expectation is taken with respect to the distribution of the input variables $X$. Then, the average squared bias is 
		\begin{align*}
			& \, \E_{\bx_0} \bracks[\big]{ \parens{ f \parens{\bx_0} - \E \bracks{\hat{f}_{\lambda} \parens{\bx_0}}}^2 } \\ 
			= & \, \E_{\bx_0} \bracks[\big]{ \parens{ f \parens{\bx_0} - \bx_0^\top \bbeta^*}^2 } + \E_{\bx_0} \bracks[\big]{ \parens{\bx_0^\top \bbeta^* - \E \bracks{\bx_0^\top \widehat{\bbeta}_{\lambda}} }^2 } \\
			= & \, \mathrm{Ave} \bracks{ \text{Model Bias} }^2 + \mathrm{Ave}\bracks{ \text{Estimation Bias} }^2. 
		\end{align*}
		
		\begin{itemize}
			\item The \underline{first} term is the \textit{average squared model bias}, the error between the best-fitting linear approximation and the true function; 
			\item The \underline{second} term is the \textit{average squared estimation bias}, the error between the average estimate $ \E \bracks{\bx_0^\top \widehat{\bbeta}_{\lambda} } $ and the best-fitting linear approximation. 
		\end{itemize}

		\textit{Remark.} 
		\begin{enumerate}
			\item The model bias can only be reduced by \textit{enlarging} the class of linear models to a richer collection of models, by including \textit{interactions} and \textit{transformations of the variables} in the model. 
			\item For linear models fit by \underline{ordinary least squares}, the \textit{estimation bias} is zero. For \underline{restricted fits} (such as ridge regression) it is positive, and we trade it off with the benefits of a reduced variance. 
		\end{enumerate}
		
		\item \textit{Variance:} For the \textit{variance} component, $\bh$ function above becomes 
		\begin{align*}
			\widetilde{\bh} \parens{\bx_0} := \bX \parens{\bX^\top \bX + \lambda \bI_p}^{-1} \bx_0, 
		\end{align*}
		and, hence, 
		\begin{align*}
			\var \bracks{\hat{f} \parens{\bx_0}} = \sigma_{\varepsilon}^2 \norm{\widetilde{\bh} \parens{\bx_0}}_2^2 = \sigma_{\varepsilon}^2 \bx_0^\top \parens{\bX^\top \bX + \lambda \bI_p}^{-1} \bX^\top \bX \parens{\bX^\top \bX + \lambda \bI}^{-1} \bx_0. 
		\end{align*}
	\end{itemize}
	
	\textit{Remark.} The discussion above about the finer decomposition is \emph{not} restricted to ridge regression but can be generalized to any restricted fits. 
	
\end{enumerate}


\section*{III. Optimism of the Training Error Rate} 

\begin{enumerate}[label=\textbf{\arabic*.}]
	
	\item \textbf{Generalization Error:} Given a training set $ \train := \set{ \parens{\bx_i, y_i}}_{i=1}^n$, the \emph{generalization error} of a model $\hat{f}$ is 
	\begin{align}
		\mathrm{Err}_{\train} := \E_{X^0, Y^0} \bracks[\big]{ L \parens{Y^0, \hat{f} \parens{X^0} } \,\vert\, \train }, 
	\end{align}
	where the training dataset $\train$ here is considered to be \emph{fixed} and $\parens{X^0, Y^0}$ is a new test data point drawn from the joint distribution of the data. 
	
	\item \textbf{Expected Prediction Error:} Based on the definition of generalization error above, the \emph{expected prediction error} is obtained by averaging over the training set $\train$, that is, 
	\begin{align}
		\mathrm{Err} := \E_{\train} \bracks[\big]{ \E_{X^0, Y^0} \bracks{ L \parens{Y^0, \hat{f} \parens{X^0} } \,\vert\, \train } }. 
	\end{align}
	
	\textit{Remark 1.} The quantity $\mathrm{Err}$ is more amenable to statistical analysis. 

	\textit{Remark 2.} It turns out that most methods effectively estimate the $\mathrm{Err}$ rather than $\mathrm{Err}_{\train}$. 
	
	\item \textbf{Training Error:} The \emph{training error} is defined to be 
	\begin{align}
		\overline{\mathrm{err}} := \frac{1}{n} \sum_{i=1}^n L \parens{y_i, \hat{f} \parens{\bx_i}}.  
	\end{align}
	
	\textit{Remark.} Typically, $\overline{\mathrm{err}} \le \text{Err}_{\train}$, since the same data $\train$ is used to fit the method and assess its error, and the model typically adapts to the training set, resulting in an \textit{overly optimistic} estimate of the generalization error $\mathrm{Err}_{\train}$. 
	
	\item \textbf{Extra- and In- Sample Error:} The quantity $\mathrm{Err}_{\train}$ can be thought of as \textit{extra-sample error} since the new test point $\parens{X^0, Y^0}$ does \emph{not} coincide the training input vectors. \\ The \textit{in-sample error} is defined to be 
	\begin{align}
		\mathrm{Err}_{\mathrm{in}} = \frac{1}{n} \sum_{i=1}^n \E_{\bY^0} \bracks[\big]{ L \parens{Y^0_i, \hat{f} \parens{\bx_i} } \,\vert\, \train }, 
	\end{align}
	where $\bY^0 := \parens{Y_1^0, \cdots, Y_n^0}^\top \in \Real^n$ indicates that we observe $n$ \emph{new} response values at each of the training points $\bx_i$, for all $i = 1, \cdots, n$. 
	
	\item \textbf{Optimism:} The \textit{optimism} is defined to be the difference between $\mathrm{Err}_{\mathrm{in}}$ and the training error $\overline{\mathrm{err}}$, 
	\begin{align*}
		\mathrm{op} := \mathrm{Err}_{\mathrm{in}} - \overline{\mathrm{err}}. 
	\end{align*}
	
	\textit{Remark.} The optimism is typically positive as $\overline{\mathrm{err}}$ is usually biased downward as an estimate of prediction error. 
	
	\item \textbf{Average Optimism:} The \textit{average optimism} is defined to be the expectation of the optimism over the training sets 
	\begin{align}
		\omega := \E_{\bY} \bracks{\mathrm{op}}, 
	\end{align}
	where $\bY := \parens{y_1, y_2, \cdots, y_n}^\top \in \Real^n$ and the predictors in the training dataset are viewed as fixed and the expectation is taken over the \emph{training set outcome values}. 
	
	\item \textbf{Average Optimism for Squared Error Loss:} When the loss function is the squared error loss function, we have 
	\begin{align*}
		\mathrm{Err}_{\mathrm{in}} = & \, \frac{1}{n} \sum_{i=1}^n \E_{\bY^0} \bracks{\parens{Y_i^0 - \hat{f} \parens{\bx_i}}^2}, \qquad \text{ and } \qquad 
		\overline{\mathrm{err}} = \frac{1}{n} \sum_{i=1}^n \parens{y_i - \hat{f} \parens{\bx_i}}^2. 
	\end{align*}
	We show that the average optimism is 
	\begin{align}\label{eq-ave-op}
		\omega = \frac{2}{n} \sum_{i=1}^n \cov \parens{y_i, \hat{y_i}}. 
	\end{align}
	where $\cov$ indicates the covariance. 
	
	First note the following 
	\begin{align*}
		\mathrm{Err}_{\mathrm{in}} = & \, \frac{1}{n} \sum_{i=1}^n \E_{\bY^0} \bracks[\big]{\parens{Y_i^0 - f \parens{\bx_i} + f \parens{\bx_i} - \E \bracks{\hat{f} \parens{\bx_i}} + \E \bracks{\hat{f} \parens{\bx_i}} - \hat{f} \parens{\bx_i}}^2} \\ 
		= & \, \frac{1}{n} \sum_{i=1}^n \E_{\bY^0} \bigg[ 
		\parens{Y_i^0 - f \parens{\bx_i} }^2 + \parens{ f \parens{\bx_i} - \E \bracks{\hat{f} \parens{\bx_i}} }^2 + \parens{\E \bracks{\hat{f} \parens{\bx_i}} - \hat{f} \parens{\bx_i}}^2 \\ 
		& \qquad + 2 \parens{Y_i^0 - f \parens{\bx_i}} \parens{ f \parens{\bx_i} - \E \bracks{\hat{f} \parens{\bx_i}}} + 2 \parens{Y_i^0 - f \parens{\bx_i} } \parens{ \E \bracks{\hat{f} \parens{\bx_i}} - \hat{f} \parens{\bx_i}} \\ 
		& \qquad \qquad + 2 \parens{f \parens{\bx_i} - \E \bracks{\hat{f} \parens{\bx_i}} } \parens{\E \bracks{\hat{f} \parens{\bx_i}} - \hat{f} \parens{\bx_i}}
		\bigg] \\ 
		= & \, \frac{1}{n} \sum_{i=1}^n \bigg[ \sigma_{\varepsilon}^2 + \parens{ f \parens{\bx_i} - \E \bracks{\hat{f} \parens{\bx_i}} }^2 + \parens{\E \bracks{\hat{f} \parens{\bx_i}} - \hat{f} \parens{\bx_i}}^2 \\ 
		& \qquad + 2 \parens{f \parens{\bx_i} - \E \bracks{\hat{f} \parens{\bx_i}} } \parens{\E \bracks{\hat{f} \parens{\bx_i}} - \hat{f} \parens{\bx_i}} \bigg]. 
	\end{align*}
	Also, note the following 
	\begin{align*}
		\overline{\mathrm{err}} = & \, \frac{1}{n}\sum_{i=1}^n \parens[\big]{y_i - f \parens{\bx_i} + f \parens{\bx_i} - \E \bracks{\hat{f} \parens{\bx_i}} + \E \bracks{\hat{f} \parens{\bx_i}} - \hat{f} \parens{\bx_i}}^2 \\ 
		= & \, \frac{1}{n} \sum_{i=1}^n \bigg[ \parens{y_i - f \parens{\bx_i} }^2 + \parens{ f \parens{\bx_i} - \E \bracks{\hat{f} \parens{\bx_i}} }^2 + \parens{\E \bracks{\hat{f} \parens{\bx_i}} - \hat{f} \parens{\bx_i}}^2 \\ 
		& \qquad + 2 \parens{y_i - f \parens{\bx_i}} \parens{ f \parens{\bx_i} - \E \bracks{\hat{f} \parens{\bx_i}}} + 2 \parens{y_i - f \parens{\bx_i} } \parens{ \E \bracks{\hat{f} \parens{\bx_i}} - \hat{f} \parens{\bx_i}} \\ 
		& \qquad \qquad + 2 \parens{f \parens{\bx_i} - \E \bracks{\hat{f} \parens{\bx_i}} } \parens{\E \bracks{\hat{f} \parens{\bx_i}} - \hat{f} \parens{\bx_i}}^2
		\bigg] \\ 
		= & \, \frac{1}{n} \sum_{i=1}^n \bigg[ \varepsilon_i^2 + \parens{ f \parens{\bx_i} - \E \bracks{\hat{f} \parens{\bx_i}} }^2 + \parens{\E \bracks{\hat{f} \parens{\bx_i}} - \hat{f} \parens{\bx_i}}^2 \\ 
		& \qquad + 2 \varepsilon_i \parens{ f \parens{\bx_i} - \E \bracks{\hat{f} \parens{\bx_i}}} + 2 \varepsilon_i \parens{ \E \bracks{\hat{f} \parens{\bx_i}} - \hat{f} \parens{\bx_i}} \\ 
		& \qquad \qquad + 2 \parens{f \parens{\bx_i} - \E \bracks{\hat{f} \parens{\bx_i}} } \parens{\E \bracks{\hat{f} \parens{\bx_i}} - \hat{f} \parens{\bx_i}}^2
		\bigg]. 
	\end{align*}
	Therefore, the optimism is given by 
	\begin{align*}
		\mathrm{op} = & \, \mathrm{Err}_{\mathrm{in}} - \overline{\mathrm{err}} \\ 
		= & \, \sigma_{\varepsilon}^2 - \frac{1}{n} \sum_{i=1}^n \varepsilon_i^2 - \frac{2}{n} \sum_{i=1}^n \varepsilon_i \parens{ f \parens{\bx_i} - \E \bracks{\hat{f} \parens{\bx_i}}} - \frac{2}{n} \sum_{i=1}^n \varepsilon_i \parens{ \E \bracks{\hat{f} \parens{\bx_i}} - \hat{f} \parens{\bx_i}}. 
	\end{align*}
	As a consequence, the average optimism is 
	\begin{align*}
		\E_{\bY} \bracks{\mathrm{op}} = & \, - \frac{2}{n} \sum_{i=1}^n \E_{\bY} \bracks[\big]{\varepsilon_i \parens{ \E \bracks{\hat{f} \parens{\bx_i}} - \hat{f} \parens{\bx_i}}} \\ 
		= & \, \frac{2}{n} \sum_{i=1}^n \E_{\bY} \bracks[\big]{ \parens{y_i - \E \bracks{y_i}} \parens{ \hat{f} \parens{\bx_i}} - \E \bracks{\hat{f} \parens{\bx_i}}} \\ 
		= & \, \frac{2}{n} \sum_{i=1}^n \cov \parens{y_i, \hat{y_i}}. 
	\end{align*}
	
	\textit{Remark 1.} For 0-1 and some other loss functions, we also have the average optimism is given by \eqref{eq-ave-op}. 
	
	\textit{Remark 2.} Equation \eqref{eq-ave-op} suggests that the amount by which $\overline{\mathrm{err}}$ underestimates the true error depends on how strongly $y_i$ affects its own prediction. 
	
	\item \textbf{Relationship Among In-sample Error, Training Error and Average Optimism:} Combining all results above, we have 
	\begin{equation}\label{eq-relas}
		\E_{\bY} \bracks{\mathrm{Err}_{\mathrm{in}}} = \E_{\bY} \bracks{\overline{\mathrm{err}}} + \frac{2}{n} \sum_{i=1}^n \cov \parens{\hat{y_i}, y_i}. 
	\end{equation}
	
	\textit{Remark.} Suppose we work with linear regression and $\hat{y_i}$ is obtained by a linear fit with $p$ inputs or basis functions, for example, for the additive error model $Y = f \parens{X} + \varepsilon$, we have 
	\begin{align*}
		\sum_{i=1}^n \cov \parens{\hat{y_i}, y_i} = p \sigma_{\varepsilon}^2, 
	\end{align*}
	and \eqref{eq-relas} simplfies to 
	\begin{align}\label{eq-insample}
		\E_{\bY} \bracks{\mathrm{Err}_{\mathrm{in}}} = \E_{\bY} \bracks{\overline{\mathrm{err}}} + \frac{2p}{n} \sigma_{\varepsilon}^2. 
	\end{align}
	This result says that the optimism increases linearly with the number of inputs or basis functions, $p$, and decreases with the training sample size. 
	
	\item \textbf{Approaches to Estimation of the Prediction Error:} 
	\begin{enumerate}
		\item \textit{Approach 1:} Estimate the optimism and add it to the training error $\overline{\mathrm{err}}$. Examples include $C_p$, AIC, BIC; 
		\item \textit{Approach 2:} Estimate the extra-sample error $\mathrm{Err}$ directly. Examples include cross-validation and bootstrap. 
	\end{enumerate}
\end{enumerate}


\section*{IV. Estimates of In-Sample Prediction Error}

\begin{enumerate}[label=\textbf{\arabic*.}]

	\item \textbf{In-Sample Error Estimate:} The general form of the in-sample error estimate is 
	\begin{align*}
		\widehat{\mathrm{Err}}_{\mathrm{in}} = \overline{\mathrm{err}} + \hat{\omega}, 
	\end{align*}
	where $\hat{\omega}$ is an estimate of the average optimism. 
	
	\item \textbf{Mallow's $C_p$ Statistic:} Based on the expression \eqref{eq-insample}, where $p$ parameters are fit under the squared error loss function, we obtain the $C_p$ statistic as
	\begin{align}
		C_p = \overline{\mathrm{err}} + \frac{2p}{n} \hat{\sigma}_{\varepsilon}^2, 
	\end{align}
	where $\hat{\sigma}_{\varepsilon}^2$ is an estimate of the variance. 
	
	\item \textbf{Akaike Information Criterion (AIC):} The \textit{Akaike Information Criterion} (AIC) is a more general estimate of $\mathrm{Err}_{\mathrm{in}}$ when a log-likelihood loss function is used. 
	\begin{enumerate}
		\item \textit{Theoretical Basis:} The development of AIC relies on the asymptotic result 
		\begin{align*}
			-2 \E \bracks{ \log \Pr_{\hat{\theta} } \parens{Y}} \approx - \frac{2}{n} \E \bracks{\text{log-likelihood}} + \frac{2p}{n}, 
		\end{align*}
		where $ \Pr_{\theta} $ is a family of densities for $Y$ containing the ``true'' density of it, $\hat{\theta}$ is the maximum-likelihood estimate of $\theta$, and ``log-likelihood'' is the maximized log-likelihood given by 
		\begin{align*}
			\text{log-likelihood} = \sum_{i=1}^n \log \Pr_{\hat{\theta}}\parens{y_i}. 
		\end{align*}
		
		\item \textit{Examples:}
		\begin{itemize}
			\item For the \textit{logistic regression} using the binomial log-likelihood, we have 
			\begin{align}
				\mathrm{AIC} = - \frac{2}{n} \text{log-likelihood} + \frac{2p}{n}. 
			\end{align}
			\item For the \textit{Gaussian regression} with variance $\sigma_{\varepsilon}^2 = \hat{\sigma}_{\varepsilon}^2$ assumed to be \emph{known}, the {AIC} statistics is equivalent to $C_p$. 
		\end{itemize}
		
		\item \textit{Using AIC for Model Selection:} To use the AIC for model selection, we choose the one with the \emph{smallest} AIC over the set of models considered. 
		
	\end{enumerate}
	
	\item \textbf{AIC for Model Selection in General Setting:} For nonlinear and more complex models, we need to replace $p$ appearing in AIC by some measure of model complexity. \newline Given a family of models ${f}_{\alpha}$ indexed by the tuning parameter $\alpha$, we let $\overline{\mathrm{err}} \parens{\alpha}$ be the training error and $p \parens{\alpha}$ be the number of parameters for each model. Then, the AIC, depending on $\alpha$, is defined to be 
	\begin{align}
		\mathrm{AIC} \parens{\alpha} = \overline{\mathrm{err}} \parens{\alpha} + \frac{2 p \parens{\alpha}}{n} \hat{\sigma}_{\varepsilon}^2, 
	\end{align}
	which provides an estimate of the \textit{test error curve}. We find the tuning parameter $\hat{\alpha}$ that minimizes $\mathrm{AIC} \parens{\alpha}$ and the final model is $f_{\hat{\alpha}}$. 
	
	% \textit{Note:} Suppose we have a total of $p$ inputs and we choose the best-fitting linear model with $d < p$ inputs, the optimism will exceed $2\cdot \frac{d}{N}\sigma_\epsilon^2$, i.e., the effective number of parameters fit is more than $d$. 
	
	\item \textbf{A Remark:} The formula 
	\begin{align*}
		\frac{2}{n} \sum_{i=1}^n \cov \parens{\hat{y_i}, y_i} = \frac{2p}{n}\sigma_{\varepsilon}^2
	\end{align*}
	\emph{exactly} holds for linear models with additive errors and squared error loss, and \emph{approximately} for linear models and log-likelihoods. 
	
	In particular, it does \emph{not} hold in general for the 0-1 loss function. 
	
\end{enumerate}


\section*{V. The Effective Number of Parameters}

\begin{enumerate}[label=\textbf{\arabic*.}]

	\item \textbf{Overview:} We generalize the concept of the ``number of parameters'', especially in the context of regularized model fitting. 
	
	\item \textbf{Notation:} We let 
	\begin{align*}
		\bY = 
		\begin{pmatrix} 
			\  y_1 & y_2 & \cdots & y_n \  
		\end{pmatrix}^\top \in \Real^n 
	\end{align*}
	to be the vector of responses and 
	\begin{align*}
		\widehat{\bY} = 
		\begin{pmatrix} 
			\  \hat{y}_1 & \hat{y}_2 & \cdots & \hat{y}_n \ 
		\end{pmatrix}^\top \in \Real^n
	\end{align*}
	be the vector of the fitted values. 
	
	\item \textbf{Linear Fitting Method:} A method is said to a \emph{linear fitting method} if we can write 
	\begin{align*}
		\widehat{\bY} = \bS \bY, 
	\end{align*}
	where $\bS \in \Real^{n \times n}$ is a matrix depending \emph{not} on the input vectors $\bx_i$'s but \emph{not} on $y_i$'s. 
	
	\item \textbf{Effective Number of Parameters for Linear Fitting Methods:} The \textit{effective number of parameters} for linear fitting methods is defined to be 
	\begin{align*}
		\mathrm{df} \parens{\bS} = \tr \parens{\bS} = \sum_{i=1}^n \bS_{i,i}, 
	\end{align*}
	where $\bS_{i,i}$ is the $i$-th diagonal element of $\bS$. The quantity $\mathrm{df}$ is also known as the \textit{effective degrees-of-freedom}. 
	
	\item \textbf{Examples:} 
	\begin{itemize}
		\item If $\bS$ is an \emph{orthogonal-projection matrix} onto a basis set spanned by $M$ features, then $ \tr \parens{\bS} = M $; 
		\item If $\bY$ arises from an additive-error model 
		\begin{align*}
			Y = f \parens{X} + \varepsilon, \qquad \text{ where } \var \bracks{\varepsilon} = \sigma_{\varepsilon}^2, 
		\end{align*}
		then 
		\begin{align*}
			\sum_{i=1}^n \cov \parens{\hat{y}_i, y_i} = \tr \parens{\bS} \sigma_{\varepsilon}^2, 
		\end{align*}
		which motivates the more general definition 
		\begin{align*}
			\mathrm{df} \parens{\widehat{\bY}} = \frac{1}{\sigma_{\varepsilon}^2} \sum_{i=1}^n \cov \parens{\hat{y_i}, y_i}; 
		\end{align*}
		
		\item In the context of neural networks where we minimize an error function $R \parens{\bw}$ with the weight decay penalty $\alpha \bw^\top \bw = \alpha \sum_{m=1}^M w_m^2$, the effective number of parameters is of the form 
		\begin{align*}
			\mathrm{df} \parens{\alpha} = \sum_{m=1}^M \frac{\theta_m}{\theta_m + \alpha}, 
		\end{align*}
		where $\theta_m$'s are the eigenvalues of $\dfrac{\partial^2 R \parens{\bw}}{\partial \bw \partial \bw^\top}$. 
	\end{itemize}
	
\end{enumerate}


\section*{VI. The Bayesian Approach and BIC}

\begin{enumerate}[label=\textbf{\arabic*.}]

	\item \textbf{Introduction to BIC:} The \emph{Bayesian Information Criterion (BIC)} is applicable when the fitting is carried out by maximization of a log-likelihood function, which has the general form 
	\begin{align}\label{eq-bic-def}
		\mathrm{BIC} = -2 \times \text{log-likelihood} + p \log n. 
	\end{align}
	
	\item \textbf{BIC under the Gaussian Model:} Under the Gaussian model and the assumption that $\sigma_{\varepsilon}^2$ is known, we have 
	\begin{align*}
		-2 \times \text{log-likelihood} = & \, \frac{1}{\sigma_{\varepsilon}^2} \sum_{i=1}^n \parens{y_i - \hat{f} \parens{\bx_i}}^2 \qquad \text{up to a constant} \\ 
		= & \frac{n}{\sigma_{\varepsilon}^2} \overline{\mathrm{err}}. 
	\end{align*}
	Hence, we can write 
	\begin{align}
		\mathrm{BIC} = \frac{n}{\sigma_{\varepsilon}^2} \bracks[\bigg]{ \overline{\mathrm{err}} + \parens{\log n} \frac{p}{n}\sigma_{\varepsilon}^2 }, 
	\end{align}
	which is proportional to the AIC with the factor $2$ replaced by $\log n$. Therefore, BIC tends to penalize complex models \emph{more heavily} and gives preference to simpler models. 
	
	\item \textbf{Derivation of BIC:} The BIC is derived from a Bayesian approach to model selection. Assume that 
	\begin{itemize}
		\item we have a set of candidate models $\calM_m$, for $ m = 1, \cdots, M$, with the corresponding model parameters $ \theta_m $, and we wish to choose a best model from them; and 
		\item we have a prior distribution $\Pr \parens{\theta_m \,\vert\, \calM_m} $ for the parameters in each model $\calM_m$. 
	\end{itemize}
	Then, letting $\bZ := \sets{\parens{\bx_i, y_i}}_{i=1}^n$ be the training data and applying Bayes Theorem, the posterior probability of a given model is 
	\begin{equation*}
		\begin{aligned}
			\Pr \parens{\calM_m \,\vert\, \bZ} \propto & \, \Pr \parens{\calM_m} \cdot \Pr \parens{\bZ \,\vert\, \calM_m} \\ 
			\propto & \, \Pr \parens{\calM_m} \int \Pr \parens{\bZ \,\vert\, \calM_m, \theta_m} \parens{\theta_m \,\vert\, \calM_m} \, \diff \theta_m. 
		\end{aligned}
	\end{equation*}
	To compare two distinct models $\calM_m$ and $\calM_l$, with $m \neq l$, we take the ratio of their posterior probabilities
	\begin{equation}\label{bf}
		\calR := \frac{\Pr \parens{\calM_m \,\vert\, \bZ}}{\Pr \parens{\calM_l \,\vert\, \bZ}} = \frac{\Pr \parens{\calM_m}}{\Pr \parens{\calM_l}} \cdot \frac{\Pr \parens{\bZ \,\vert\, \calM_m}}{\Pr \parens{\bZ \,\vert\, \calM_l}}. 
	\end{equation}
	We choose $\calM_m$ if $\calR > 1$ and $\calM_l$ otherwise. 
	
	\item \textbf{Bayes Factor:} The quantity appearing in \eqref{bf} 
	\begin{align}
		\mathrm{BF} \parens{\bZ} := \frac{\Pr \parens{\bZ \,\vert\, \calM_m}}{\Pr \parens{\bZ \,\vert\, \calM_l}}
	\end{align}
	is called the \emph{Bayes Factor}, which is interpreted as ``the contribution of the data toward the posterior odds''. 

	\item \textbf{Laplace Approximation of $\Pr \parens{\bZ \,\vert\, \calM_m}$:} Typically, one assumes that the prior probability distribution over the models $ \sets{\mathcal{M}_m}_{m=1}^M $ is uniform so that $\Pr \parens{\calM_m} = \frac{1}{M}$ is constant for all $m = 1, 2, \cdots, M$. And one needs to approximate $\Pr \parens{\bZ \,\vert\, \calM_m}$. One way to perform this approximation is the \textit{Laplace approximation} 
	\begin{equation}\label{eq-bic-deriv}
		\log \Pr \parens{\bZ \,\vert\, \calM_m} = \log \parens{\bZ \,\vert\, \hat{\theta}_m, \calM_m} - \frac{p_m}{2} \log n + \calO \parens{1}, 
	\end{equation}
	where $ \hat{\theta}_m $ is a maximum likelihood estimate and $p_m$ is the number of free parameters in model $ \calM_m$. 
	
	If we let the loss function be 
	\begin{align*}
		-2 \times \log \Pr \parens{\bZ \,\vert\, \hat{\theta}_m, \calM_m}, 
	\end{align*}
	the resulting quantity is equivalent to \eqref{eq-bic-def}. 
	
	\item \textbf{BIC for Model Selection:} We choose the model $\calM_{m^*}$ that gives the smallest BIC among all $\calM_1, \cdots, \calM_M$. \newline From \eqref{eq-bic-deriv}, we see that choosing the model with the \textit{minimum BIC} is equivalent to choosing the model with the largest (approximate) posterior probability. 
	
	\item \textbf{Posterior Probabilities for $\sets{\calM_m}_{m=1}^M$:} If one is able to compute the BIC criterion for a set of $M$ models, denoted by $\mathrm{BIC}_m$, for all $ m = 1, \cdots, M$, one can estimate the posterior probability of model $\calM_m$ as 
	\begin{align*}
		\frac{\exp \parens{- \frac{1}{2} \mathrm{BIC}_m}}{\sum_{l=1}^m \exp \parens{-\frac{1}{2} \mathrm{BIC}_l}}. 
	\end{align*}
	
	\item \textbf{A Comparison between AIC and BIC:} 
	\begin{itemize}
		\item For model selection purposes, there is no clear choice between AIC and BIC; 
		\item BIC is asymptotically consistent as a selection criterion in the sense that ``\textit{given a family of models, including the true model, the probability that BIC will select the correct model approaches one as the sample size $n \to \infty$}'', and AIC does \emph{not} have this property and tends to choose too complex models as $n \to \infty$; 
		\item For finite samples, BIC tends to choose models that are \emph{too} simple, due to the heavy penalty on complexity. 
	\end{itemize}
\end{enumerate}


%\section*{VIII. Vapnik-Chervonenkis Dimension:}
%
%The Vapnik–Chervonenkis (VC) theory provides \textit{a general measure of complexity} and gives \textit{associated bounds on the optimism}. Throughout this section, we use the notation $\left\{ f\parens{\bx, \balpha} \right\}$ to denote a family of functions with the input variable $\bx$ and the parameter $\balpha$. 
%
%\begin{enumerate} 
%\item \textbf{A Motivating Example --- Indicator Function Case:} Assume that $f$ is an indicator function only taking on values of 0 and 1. 
%\begin{itemize}
%	\item Further, assume $\balpha = \parens{\alpha_0, \balpha_1}$ and $f$ is a linear indicator function of the form $f\parens{\bx, \balpha} = \indic\parens{\alpha_0 + \alpha_1^\top \bx > 0}$. It is reasonable to say the complexity of the class $f$ is the number of parameters $ p + 1 $. 
%	\item Now, let $\bx = x \in \mathbb{R}$ and $\balpha = \alpha \in \mathbb{R}$ and consider the function of the form $f\parens{x, \alpha} = \indic\parens{\sin\parens{\alpha \cdot x} > 0}$. As $\alpha$ increases, the function becomes more wiggly and more complex, even though we only have one parameter in $\mathbb{R}$. 
%	\begin{figure}[h]
%	    \centering
%		\includegraphics[scale = 0.5]{plot_7_5}
%		\caption{$f\parens{x} = \sin\parens{\alpha\cdot x}$.}
%	\end{figure}
%\end{itemize} 
%
%\item \textbf{Vapnik-Chervonenkis Dimension:} The \textit{Vapnik - Chervonenkis dimension} is a way of measuring the \underline{complexity of a class of functions} by assessing how wiggly its members can be. More formally, the VC dimension of the class $\left\{ f\parens{\bx, \balpha} \right\}$ is defined to be the largest number of points (in some configuration) that can be shattered by members of $\left\{ f\parens{\bx, \balpha} \right\}$. 
%\end{enumerate}


\section*{VII. Cross-validation}

\begin{enumerate}[label=\textbf{\arabic*.}]
	
	\item \textbf{Overview:} \emph{Cross-validation} directly estimates the \textit{expected extra-sample error} 
	\begin{align*}
		\mathrm{Err} = \E \bracks{L \parens{Y, \hat{f} \parens{X}}}, 
	\end{align*}
	the average generalization error when $\hat{f}$ is applied to an independent test sample from the joint distribution of $X$ and $Y$. 
	
	\item \textbf{$K$-Fold Cross-Validation:} 
	\begin{enumerate}
		\item \textit{Motivation:} Ideally, when we have sufficient data, we set aside a validation set and use it to assess the performance of the prediction model. Since data are often scarce, this is \emph{not} usually possible. 
		\item \textit{Main Idea:} The $K$-fold cross-validation uses part of data to fit the model and a different set to assess the model. 
		\item \textit{Procedure:} 
		\begin{itemize}
			\item We split the dataset into $K$ roughly equal-sized parts. $K$-fold cross-validation uses $\parens{K-1}$ parts, excluding the $k$-th part, to fit the model and the remaining $k$-th part to test it. With the remaining part of data, we calculate the prediction error of the fitted model. We do this for $k = 1, 2, \cdots, K$ and combine the $K$ estimates of the prediction error; 
			\item Let $\kappa: \sets{ 1, \cdots, n } \to \sets{ 1, \cdots, K }$ be an indexing function that indicates the partition to which observation $i$ is allocated by \textit{randomization}. Let $\hat{f}^{-k}$ denote the fitted function, computed without the $k$-th part of data. The cross-validation estimate of the prediction error is 
			\begin{equation}
				\cv \parens{\hat{f}} = \frac{1}{n} \sum_{i=1}^n L \parens{y_i, \hat{f}^{-\kappa \parens{i}} \parens{\bx_i}}. 
			\end{equation}
		\end{itemize} 
		\item \textit{Remarks:}
		\begin{itemize}
			\item Typical choices of $K$ are 5 or 10. 
			\item The case when $ K = n $ is known as \textit{leave-one-out cross-validation}, and in this case, $\kappa \parens{i} = i$. 
		\end{itemize}

	\end{enumerate}
	
	\item \textbf{Using $K$-Fold CV for Model Selection:} Consider a set of models $\sets{ f_{\alpha} }_{\alpha}$ with $\bx$ being the predictors and $\alpha$ being the tuning parameter. Let $\hat{f}_{\alpha}^{-k}$ be the $\alpha$-th model fit with the $k$-th part of the data removed. Then, the following quantity 
	\begin{align}
		\cv \parens{\hat{f}_{\alpha}} := \frac{1}{n} \sum_{i=1}^n L \parens{y_i, \hat{f}_{\alpha}^{-\kappa \parens{i}} \parens{\bx_i}}
	\end{align}
	provides an estimate of the \emph{test error curve} (as a function of $\alpha$). We choose the value of $\alpha$ that minimizes $\cv \parens{\hat{f}_{\alpha}}$, say $\hat{\alpha}$. The final model is $f_{\hat{\alpha}}$. 
	
	\item \textbf{Discussion on Choice of $K$ in $K$-Fold Cross-Validation:} 
	\begin{enumerate}
		\item With $\underline{K = n}$, 
		\begin{enumerate}
			\item the cross-validation estimator is approximately unbiased for the true \emph{expected predictor error}, but can have \emph{high variance}; 
			\item the computation burden is considerable. 
		\end{enumerate}
		
		\item With a \underline{smaller value of $K$}, say $K = 5$ or $10$, 
		\begin{enumerate}
			\item cross-validation has \emph{lower variance} but may have \emph{large bias}, which depends on how the performance of the learning method varies with the size of the training set; 
			\item if the learning curve (plotting cross-validation error against size of training set) has a considerable slope at the given training set size, $5$- or $10$-fold cross-validation may overestimate the true prediction error. 
		\end{enumerate}
	\end{enumerate}
	
	\textit{Conclusion:} 5- or 10-fold cross-validation is recommended. 
	
	\item \textbf{Generalized Cross-Validation:} \emph{Generalized cross-validation (GCV)} provides a convenient approximation to leave-one-out cross-validation for \emph{linear fitting} under the \textit{squared-error loss function}. 
	\begin{enumerate}
		\item \textit{Review of linear fitting methods:} A linear fitting method is the one for which we can write 
		\begin{align*}
			\widehat{\bY} = \bS \bY, 
		\end{align*}
		where $\bY := \parens{y_1, \cdots, y_n}^\top \in \Real^n$ is the vector of the observed response variable, $\widehat{\bY} := \parens{\hat{y}_1, \cdots, \hat{y}_n}^\top \in \Real^n$ is the vector of the fitted values, and $\bS \in \Real^{n \times n}$. 
		
		\item \textit{Leave-one-out cross-validation relationship:} For many linear fitting methods, including the least squares method and cubic smoothing splines, we have 
		\begin{align*}
			\frac{1}{n} \sum_{i=1}^n \parens{y_i - \hat{f}^{-i} \parens{\bx_i}}^2 = \frac{1}{n} \sum_{i=1}^n \parens[\bigg]{\frac{y_i - \hat{f}\parens{\bx_i}}{1 - \bS_{i,i}}}^2, 
		\end{align*}
		where $\bS_{i,i}$ is the $i$-th diagonal element of the matrix $\bS$. 
		
		\item \textit{GCV approximation:} The \emph{GCV approximation} is defined to be 
		\begin{align}\label{eq-gcv}
			\mathrm{GCV} \parens{\hat{f}} := \frac{1}{n} \sum_{i=1}^n \parens[\bigg]{\frac{y_i - \hat{f} \parens{\bx_i}}{1 - \tr \parens{\bS}/n}}^2, 
		\end{align}
		where the quantity $\tr\parens{\bS}$ is the effective number of parameters. 
		
		\item \textit{Comparison between GCV and CV:} The GCV has the computing advantage in the settings where the trace of $\bS$ is easier to compute than the individual elements of $\bS$. 

		\item \textit{Connection between GCV and $C_p$ and AIC:} Using the approximation $\frac{1}{\parens{1-x}^2} \approx 1 + 2x$, we have 
		\begin{align}
			\mathrm{GCV} \approx & \, \frac{1}{n} \sum_{i=1}^n \parens{y_i - \hat{f} \parens{\bx_i}}^2 \parens[\bigg]{ 1 + \frac{2 \tr \parens{\bS}}{n}} \nonumber \\ 
			= & \, \frac{1}{n} \sum_{i=1}^n \parens{y_i - \hat{f} \parens{\bx_i}}^2 + \frac{2 \tr \parens{\bS}}{n} \parens[\bigg]{\frac{1}{n} \sum_{i=1}^n \parens{y_i - \hat{f} \parens{\bx_i}}^2}. \label{eq-gcv2}
		\end{align}
		In \eqref{eq-gcv2}, the first term $\frac{1}{n} \sum_{i=1}^n \parens{y_i - \hat{f} \parens{\bx_i}}^2$ is the training error, $\overline{\mathrm{err}}$, when the squared-error loss function is used. In the second term, $\hat{\sigma}_{\varepsilon}^2 \approx \frac{1}{n} \sum_{i=1}^n \parens{y_i - \hat{f} \parens{\bx_i}}^2$, and $\tr \parens{\bS} = p$. Combining all pieces together, we see 
		\begin{align*}
			\mathrm{GCV} \approx C_p. 
		\end{align*}
		Assume the random error term has a Gaussian distribution and $\var \bracks{\varepsilon} = \sigma_{\varepsilon}^2$ is known, we have $\mathrm{GCV} \approx \sigma_{\varepsilon}^2 \times \mathrm{AIC}$. 
	\end{enumerate}
	
	\item \textbf{Correct Way of Doing Cross-Validation in a Multistep Modeling Procedures:} In a multistep modeling procedure, cross-validation must be applied to the \emph{entire} sequence of modeling steps; in particular, samples must be ``left out'' \textbf{before} any selection or filtering steps are applied. 
	
%\item Consider a classification problem with a large number of predictors	
%\begin{enumerate}
% \item Divide the samples into $K$ cross-validation folds (groups) at random; 
% \item For each fold $ k = 1, 2, \cdots, K$: 
%  \begin{enumerate} 
%    \item Find a subset of ``good'' predictors that show fairly strong (univariate) correlation with the class labels, using all of the samples except those in the fold $k$; 
%    \item Using just this subset of predictors, build a multivariate classifier, using all of the samples except those in the fold $k$; 
%    \item Use the classifier to predict the class labels for the samples in fold $k$. 
%\end{enumerate}
%\end{enumerate}
\end{enumerate}


\section*{VIII. Bootstrap Methods}

\begin{enumerate}[label=\textbf{\arabic*.}]

	\item \textbf{Overview:} \textit{Bootstrap method} is a general tool for assessing statistical accuracy, and can be use to estimate extra-sample prediction error. It can estimate well the \emph{expected prediction error}. 
	
	\item \textbf{Setup:} Let $\bZ := \sets{z_i}_{i=1}^n$ denote the training sets, where $z_i := \parens{\bx_i, y_i}$ for all $i = 1, \cdots, n$. 
	
	\item \textbf{Procedures:} 
	\begin{enumerate}
		\item \emph{Randomly} draw datasets \emph{with replacement} from $\bZ$, each sample the same size as the original training set; 
		\item Repeat the preceding step for $B$ times, producing $B$ bootstrap datasets; 
		\item Refit the model to each of the bootstrap datasets, and examine the behavior of the fits over the $B$ replications; 
		\item Suppose $S \parens{\bZ}$ is some quantity of interest computed from the data $\bZ$. Compute $S \parens{\bZ^b}$ for all $b = 1, \cdots, B$, where $\bZ^b$ denotes the $b$-th bootstrapped sample. We can use $S \parens{\bZ^1}, S \parens{\bZ^2}, \cdots, S \parens{\bZ^B}$ to assess the statistical accuracy of $S \parens{\bZ}$. 
	\end{enumerate}
	
	\item \textbf{Example:} Suppose we want to estimate the variance of $S \parens{\bZ}$ using the bootstrap. The bootstrap estimate of the variance of $S \parens{\bZ}$ is 
	\begin{align*}
		\widehat{\var} \bracks{S \parens{\bZ}} = \frac{1}{B-1} \sum_{b=1}^B \parens[\big]{S \parens{\bZ^b} - \bar{S}^*}^2, 
	\end{align*}
	where $\bar{S}^* := \frac{1}{B} \sum_{b=1}^B S \parens{\bZ^b}$. 
	
	\textit{Remark.} The quantity $\widehat{\var} \bracks{S \parens{\bZ}}$ can be thought of as a \textit{Monte-Carlo estimate} of the variance of $S \parens{\bZ}$ under sampling from the empirical distribution function $\widehat{F}$ for the data $\parens{z_1, \cdots, z_n}$. 
	
	\item \textbf{Estimation of Prediction Error Using the Bootstrap --- Method 1:} Fit the model in question on a set of bootstrap samples. Let $\hat{f}^{*b} \parens{\bx_i}$ be the predicted value at $\bx_i$, from the model fitted to the $b$-th bootstrapped dataset. The estimate of the prediction error is 
	\begin{align}
		\widehat{\mathrm{Err}}_{\mathrm{boot}} := \frac{1}{B} \frac{1}{n} \sum_{b=1}^B \sum_{i=1}^n L \parens{y_i, \hat{f}^{*b} \parens{\bx_i}}. 
	\end{align}
	
	\textit{Comments:} This is a \emph{bad} estimate. Note that the training samples (the bootstrapped datasets) and the test samples (the original training dataset) have observations in common. This overlap can make overfit predictions look unrealistically good. 
	
	\item \textbf{Estimation of Prediction Error Using the Bootstrap --- Method 2:} For each observation, we only keep track of predictions from bootstrapped samples \emph{not} containing that observation. The \textit{leave-one-out bootstrap estimate of prediction error} is 
	\begin{align}\label{eq-bootstrap-2}
		\widehat{\mathrm{Err}}_{\mathrm{boot}}^{\parens{1}} = \frac{1}{n} \sum_{i=1}^n \frac{1}{\abs{C^{-i}}} \sum_{b \in C^{-i}} L \parens{y_i, \hat{f}^{*b} \parens{\bx_i}}, 
	\end{align}
	where $C^{-i}$ is the set of indices of the bootstrap samples $b$ that do \emph{not} contain the observation $i$, and $\abs{C^{-i}}$ is the number of such samples. 
	
	\textit{Remark.} In computing $\widehat{\mathrm{Err}}_{\mathrm{boot}}^{\parens{1}}$, we either 
	\begin{itemize}
		\item have to choose $B$ large enough to ensure that all of the $\abs{C^{-i}}$ are greater than zero, or 
		\item just leave out the terms in \eqref{eq-bootstrap-2} corresponding to $\abs{C^{-i}}$'s that are zero. 
	\end{itemize}
	
	\item \textbf{``0.632 Estimator'':} 
	\begin{enumerate}
		\item \textit{Motivation:} The leave-one out bootstrap estimate of the prediction error $\widehat{\mathrm{Err}}_{\mathrm{boot}}^{\parens{1}}$ has the \textit{training-set-size bias} problem discussed in the cross-validation. The average number of distinct observations in each bootstrap sample is about $0.632 \times n$, where 
		\begin{align*}
			\Pr \parens[\Big]{\text{ observation $i \in $ bootstrap samples }} = & \, 1 - \parens[\bigg]{1 - \frac{1}{n}}^n \\ 
			\approx & \, 1 - e^{-1} \\ 
			\approx & \, 0.632. 
		\end{align*}
		If the learning curve has considerable slope at sample size roughly $n/2$, the leave-one out bootstrap estimate will be biased \textit{upward} as an estimate of the true error. 
		
		\item \textit{``0.632 Estimator'':} The ``\textit{0.632 estimator}'' is designed to alleviate this upward bias and is defined to be 
		\begin{align}
			\widehat{\mathrm{Err}}_{\mathrm{boot}}^{\parens{0.632}} = 0.368 \times \overline{\mathrm{err}} + 0.632 \times \widehat{\mathrm{Err}}_{\mathrm{boot}}^{\parens{1}}. 
		\end{align}
		The \textit{main idea} is that the ``0.632 estimator'' pulls the leave-one out bootstrap estimate down toward the training error rate, and hence reduces its upward bias. 
	\end{enumerate}
	
	\item \textbf{No-information Error Rate:} The \textit{no-information error rate} is defined to be the error rate of our prediction rule as if the inputs and the outputs were independent. We denote it by $\gamma$. 
	
	\begin{enumerate}
		\item \textit{Estimation:} An estimate of $\gamma$ can be obtained by evaluating the prediction rule on \emph{all} possible combinations of targets $y_i$ and predictors $\bx_{i'}$ 
		\begin{align}
			\hat{\gamma} := \frac{1}{n^2} \sum_{i=1}^n \sum_{i'=1}^n L \parens{y_i, f \parens{\bx_{i'}}}. 
		\end{align}
		
		\item \textit{Example --- Binary classification problem:} Consider a binary classification problem with class labels being $\sets{+1, -1}$ and the loss function being the 0-1 loss. Let $\hat{p}_1$ be the observed proportion of responses equaling 1, and $\hat{q}$ be the observed proportion of predictions $f \parens{\bx_{i'}}$ equal to 1. Then, we show 
		\begin{align*}
			\hat{\gamma} = \hat{p} \parens{1 - \hat{q}} + \parens{1 - \hat{p}} \hat{q}. 
		\end{align*}
		Note the following 
		\begin{align*}
			\hat{\gamma} = & \, \frac{1}{n^2} \sum_{i=1}^n \sum_{i'=1}^n L \parens[\big]{y_i, \hat{f} \parens{\bx_{i'}} } \\ 
			= & \, \frac{1}{n^2} \sum_{i'=1}^n \sum_{\sets{i \,\vert\, y_i = +1}} \indic \parens[\big]{\hat{f} \parens{\bx_{i'}} \neq +1} + \frac{1}{n^2} \sum_{i'=1}^n \sum_{\sets{i \,\vert\, y_i = -1}} \indic \parens[\big]{\hat{f} \parens{\bx_{i'}} \neq -1} \\ 
			= & \, \frac{1}{n^2} \abs[\big]{\sets{i \,\vert\, y_i = +1}} \sum_{i'=1}^n \indic \parens[\big]{\hat{f} \parens{\bx_{i'}} \neq +1} + \frac{1}{n^2} \abs[\big]{\sets{i \,\vert\, y_i = -1}} \sum_{i'=1}^n \indic \parens[\big]{\hat{f} \parens{\bx_{i'}} \neq -1} \\ 
			= & \, \frac{1}{n^2} \abs[\big]{\sets{i \,\vert\, y_i = +1}} \cdot \abs[\big]{\sets{i' \,\vert\, \hat{f} \parens{\bx_{i'}} = -1}} + \frac{1}{n^2} \abs[\big]{\sets{i \,\vert\, y_i = -1}} \cdot \abs[\big]{\sets{i' \,\vert\, \hat{f} \parens{\bx_{i'}} = +1}} \\ 
			= & \, \hat{p} \parens{1 - \hat{q}} + \parens{1 - \hat{p}} \hat{q}. 
		\end{align*}
	\end{enumerate}
	
	\item \textbf{Relative Overfitting Rate:} The \textit{relative overfitting rate} is defined to be 
	\begin{align}
		\widehat{R} := \frac{\widehat{\mathrm{Err}}^{\parens{1}}_{\mathrm{boot}} - \overline{\mathrm{err}} }{\hat{\gamma} - \overline{\mathrm{err}} }. 
	\end{align}
	Note that $\widehat{R}$ ranges from 
	\begin{itemize}
		\item 0, if there is no overfitting and $\widehat{\mathrm{Err}}^{\parens{1}}_{\mathrm{boot}} = \overline{\mathrm{err}}$, to 
		\item 1, if the overfitting is equal to $\hat{\gamma} - \overline{\mathrm{err}}$. 
	\end{itemize}
	
	\item \textbf{``0.632+ Estimator'':} The ``\textit{0.632+ estimator}'' is defined to be 
	\begin{align}
		\widehat{\mathrm{Err}}_{\mathrm{boot}}^{\parens{0.632+}} = \parens{1 - \hat{w}} \times \overline{\mathrm{err}} + \hat{w} \times \widehat{\mathrm{Err}}_{\mathrm{boot}}^{\parens{1}}, 
	\end{align}
	where 
	\begin{align*}
		\hat{w} = \frac{0.632}{1 - 0.367 \widehat{R}}. 
	\end{align*}

	\begin{enumerate}
		\item \textit{Analysis of $\hat{w}$ and $\widehat{\mathrm{Err}}_{\mathrm{boot}}^{\parens{0.632+}}$:} Note that the weight $\hat{w}$ ranges from 0.632 if $\widehat{R} = 0$ to 1 if $\widehat{R} = 1$; and $\widehat{\mathrm{Err}}_{\mathrm{boot}}^{\parens{0.632+}}$ ranges from $\widehat{\mathrm{Err}}_{\mathrm{boot}}^{\parens{0.632}}$ to $\widehat{\mathrm{Err}}_{\mathrm{boot}}^{\parens{1}}$. 
		\item \textit{General Idea of $\widehat{\mathrm{Err}}_{\mathrm{boot}}^{\parens{0.632+}}$:} The quantity $\widehat{\mathrm{Err}}_{\mathrm{boot}}^{\parens{0.632+}}$ produces a compromise between the leave-one-out bootstrap and the training error rate that depends on the amount of overfitting. 
	\end{enumerate}
	
\end{enumerate}

\printbibliography

\end{document}
